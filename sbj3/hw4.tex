\documentclass[11pt,oneside,a4paper]{article}
\usepackage{kotex}
\usepackage{graphicx}
\usepackage{subcaption}

\begin{document}

\title{Programming Language Assignment}
\author{학번 : C019059 이름 : 신영기}
\date{\today}
\maketitle{}

\section{과제 4 : Lisp}

\noindent과목 : 프로그래밍언어론

\noindent학수번호 : 101512-002

\noindent이름 : 신영기

\noindent학번 : C019059

\noindent전공 : 화학공학

\noindent복수전공 : 컴퓨터공학

\section{Lisp 공부 내용}
\subsection{Lisp 소개}
Lisp는 컴퓨터 과학자 John McCarthy가 1958년에 창안한 언어로, 포트란에 이어 두 번째로 오래된 고급 프로그래밍 언어다.
또한 Referencial transparency를 만족시키는 함수형 프로그래밍 언어이고, 데이터 타입은 오직 원소와 리스트만 존재한다.
이런 리스프에 반복, 배열, 문자열, 복잡한 숫자 등의 기능을 더욱 추가해서 만든 언어가 common Lisp이다.
의미론, 구문론적으로 간단하고 프로그램이 자동적으로 동시성을 가지고 동작한다는 장점이 있지만 컴퓨터 구조에 중점을 두고 구현하기보다 
사람이 사고하는 방식에 중점을 두고 설계한 언어리기 때문에 효율성이 떨어진다는 단점이 있다.

\section{hw4a.lisp}
NQueen문제를 lisp를 이용해 구현하였습니다. NQueen은 대표적인 백트래킹 문제로 저는 재귀함수를 활용해 구현하였습니다.
구현 로직 : 이차원 배열을 사용하지 않고, 크기가 4인 일차원 배열을 이용해 배열의 i번 인덱스의 값이 j라면 
체스판에서 퀸의 위치를 \(x, y\)로 표현한다고 하면 \(i, j\)의 위치에 퀸이 존재한다라고 가정해서 표현하였습니다.
함수의 매개변수로 들어오는 list의 length를 구해 만약 list가 4보다 작다면 현재 페스판에 아직 4개의 퀸이 올라가지 않은 상태이므로 퀸을 추가하는 명령을 실행하였습니다.
현재 추가해야하는 체스판의 x좌표를 list의 length로 두고 y좌표를 0부터 3까지 순서대로 대입해보며 서로 겹치는 퀸이 있는지 확인하였습니다.
만약 해당 좌표에서 어떠한 퀸이랑도 겹치지 않았다면 함수호출을 위한 새 list에 매개변수로 받은 list의 뒤에 새로 넣을 queen의 좌표를 추가하여 재귀함수를 호출하였습니다.
이런 방식으로 list의 크기가 4가 될때까지 반복하고, list의 크기가 4가 되면 현재 리스트를 출력하고 함수를 종료합니다.


어려웠던 점 : setq를 사용하면 전역변수로 선언되는지 알지 못해서 새로 추가되는 퀸의 x좌표를 나타내는 $append_i$의 값이 재귀함수를 호출하면 변경되어 퀸끼리 대각선 방향으로 겹쳐서 서로를 잡을 수 있는 경우지만
이를 서로 잡을 수 없는 경우로 판단하여 정답이 2개가 아니라 4개로 출력되는 문제가 있었습니다.
반복문이 시작되기전 $append_i$의 값을 항상 동일하게 초기화하여 재귀함수에서 값이 변경되어도 호출 함수에서는 원하는 값을 사용할 수 있도록 조정하였습니다.

\subsection{구현 코드}
\begin{verbatim} 
;; 문제 접근
;; 크기가 4인 list를 만들어서 i번째 인덱스에 들어가는 값 j는 (i, j)에 queen이 들어있다는 것을 나타냄
;; 세로, (좌 + 우) 위쪽 대각선 방향만 확인
;; 가로는 어짜피 한 줄에 하나의 queen만 들어갈 수 있으므로 각 인덱스는 한 줄을 나타내도록 지정

(setq queen '())

(defun NQueen(lst)
	(if (= (length lst) 4) 
		(progn
			(loop for Q in lst
				do
				(progn 
					(princ (+ Q 1)) ;;현재 0인덱스이기 때문에 과제 설명처럼 1인덱스로 변경해서 출력
					(princ " "))
			)
			(terpri)
			(return-from NQueen nil)
		)
	)
	(loop for append_j from 0 below 4 ;; 현재 추가할 queen의 j_idx (0~3)
		do
		(progn
		(setq flag 0) ;; 다른 queen과 겹치는 경우 flag 1로 변경해서 다음 함수 호출 X
		(setq pre_i 0)
		(setq append_i (length lst)) ;; 현재 추가할 queen의 i_idx
		;; 이 append_i의 값이 전역변수라서 내부에서 함수를 호출하면 해당 호출 함수에서 append_i의 값을 바꿔 제대로된 값이 리턴되지 않았다.
		;; 처음 출력 정답 내용
		;; (2 4 1 3) 
		;; (2 4 3 1) 
		;; (3 1 4 2) 
		;; (3 2 4 1) 
		;; 처음 값은 이것처럼 대각선 값을 제대로 확인 X
		;; 모두 지역변수로 선언할 수도 있지만 매번 초기화하는 방법을 선택했습니다.
		;; append_j의 값은 함수 호출로인한 side effect로 바뀔 수 있지만, 어짜피 루프 돌면 새로운 다음 값으로 초기화돼서 로직에 영향을 끼치지 않습니다.
		(if (> append_i 0)
			(progn
			(loop for pre_j in lst ;; 비교할 queen의 i 인덱스 범위 지정
				do
				(progn
				(setq comp_i (abs (- pre_i append_i)))
				(setq comp_j (abs (- pre_j append_j))) ; i, j의 차가 0이거나 |i| == |j|면 겹치는 경우이므로 flag를 1로 설정
				(if (= comp_i comp_j) ; 대각선 확인
					(setq flag 1))
				(if (= pre_j append_j) ; 위쪽 확인
					(setq flag 1))
				(setq pre_i (+ pre_i 1))
				)
		)))
		(if (= flag 0) ;; 모든 조건을 통과했다면
			(progn
				(setq retlist (append lst (list append_j))) ;;전달용 리스트에 해당 인덱스 추가
				(NQueen retlist) ;; 재귀 호출 -> 얘는 이 list 자
			)
		)
		)
	)
)
(terpri)
(princ "NQueen size = 4")
(terpri)
(NQueen queen)
;; 배운점 : lisp의 변수 scope에 대해 생각해볼 수 있어서 좋았다.
;; call by value : 참조를 담은 값 (진짜 값이든, 포인터 변수이든 아무튼 그 변수 자체는 아님)
;; call by reference : 동일한 메모리 (&로 객체 넘기면 그거 수정하면 그 객체도 수정되듯이 진짜 변수 그자체)
;; call by == pass by 거의 같은 의미다
\end{verbatim}
\subsection{실행 사진}
\label{exPicture:hw4a}
\includegraphics[width=\textwidth]{hw4a.png}

\section{hw4b.lisp}
Insertion sort을 lisp를 이용해 구현하였습니다.
구현 로직 : 크기가 8인 리스트를 인자로 받아서 인덱스 1부터 7까지의 값을 target으로 하여 반복문(1)을 사용해서 정렬하였습니다.
반복문(2)을 돌면서 매번 임시 리스트를 뜻하는 변수인 tmplist를 매개변수로 받은 리스트로 초기화하고, 반환할 정렬된 list를 뜻하는 retlist를 생성에서 해당 인덱스의 전 인덱스까지의 값을 cdr로 뽑아내서 원하는 target값을 먼저 구했습니다.
이후 다시 tmplist를 매개변수로 받은 리스트로 초기화하고, target으로 정한 인덱스 전까지 다시 반복문(3)을 돌면서 target보다 값이 작으면 retlist에 요소를 추가하고,
target으로 정한 값보다 요소가 크다면 target값을 먼저 retlist에 추가하고 해당 요소를 retlist에 추가하는 방법으로 진행하였습니다.
반복문(3)을 마치고 target이 retlist에 들어가있다면 templist의 남은 요소에서 target 값은 제외하고 retlist와 tmplist를 이어붙였고, 
target값이 retlist에 추가되지 않았다면 retlist와 tmplist를 바로 이어붙이는 방식으로 구현했습니다.

어려웠던 점 : c++과 python같은 언어들은 자료구조를 append 또는 pop하는 경우 해당 자료구조 변수에 바로 적용이 되었는데 
lisp함수의 append는 해당 리스트를 추가한 새 리스트를 할당해서 반환하기때문에 이 부분을 인지하지 못해서 디버깅에 시간이 많이 걸렸습니다.
\subsection{구현 코드}
\begin{verbatim} 
(setq tc1 (list 11 33 23 45 13 25 8 135))
(setq tc2 (list 83 72 65 54 47 33 29 11))

(defun insertsort (lst)
	(loop for i from 1 to 7 do ; 인덱스 1부터 7까지
		(setq tmplist lst)
		(setq retlist '())
		(loop for j from 1 to i do   ; 원하는 인덱스 숫자 얻기
			(setq tmplist (cdr tmplist)))
		(setq target (car tmplist))
		; 이제 해당 숫자와 이전의 인덱스와 비교하면서 정렬 수행하기
		(setq tmplist lst)
		(setq flag 0)
		(loop for j from 0 below i do ; 내 인덱스 이전 숫자와 비교
			(setq lstnum (car tmplist))
			(setq tmplist (cdr tmplist))
			;; tmplist의 원소들을 하나씩 빼면서 target과 비교하면서 정렬
			(if (= flag 0)
				(cond
					(( < lstnum target)
				 	(setq retlist (append retlist (list lstnum)))) ;; 타겟보다 작다면 리스트의 앞에 차례로 추가
					(t
				 	(setq retlist (append retlist (list target lstnum))) ;; 타겟보다 크면 타겟을 리스트에 추가한 후 뒤에 tmplist의 원소를 추가 -> 크기순 정렬됨
				 	(setq flag 1))) ;; 이후 뒷 부분은 비교할 필요없이 retlist에 append하면 되니까 flag처리
				(setq retlist (append retlist (list lstnum))) ;;flag가 1이면 이미 target은 retlist에 들어간 상태이므로 비교하지않고 바로 append(이전 정렬에서 정렬되었음을 보장)
			);;여기서 c++와 같은 언어들은 pop을 실행하면 알아서 해당 자료구조의 크기도 변경되는데 lisp도 그럴거라고 생각하고 작성했는데 그게 아니라서 버그수정에 시간이 걸렸다.
		)
		(if (= flag 1) (setq tmplist (cdr tmplist))) ;; target이 이미 retlist에 들어있는경우 target을 tmplist에서 제외
		(setq retlist (append retlist tmplist))
		(setq lst retlist)
		;; 한 단계 끝나고 출력부분 -> 개행 없이 출력을 위해 princ활용
		(princ "step ")
		(princ i)
		(princ " : ")
		(princ retlist)
		(terpri) ;;개행 출력
	)

	(terpri)
	(princ "finish : ")
	(princ lst)
	(terpri)
)
(terpri)
(princ "Insertsort")
(terpri)(terpri)

(princ "TC1 : ")
(princ tc1)
(terpri)(terpri)

(insertsort tc1)
(setq retlist '())
(terpri)(terpri)(terpri)

(princ "TC2 : ")
(princ tc2)
(terpri)(terpri)

(insertsort tc2)
(terpri)(terpri)

\end{verbatim}
\subsection{실행 사진}
\label{myPicture:pic}
\includegraphics[width=\textwidth]{hw4b.png}

\end{document}
